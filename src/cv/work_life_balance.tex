% Adapted from https://tex.stackexchange.com/questions/82727/create-a-ring-diagram-in-tex/82729#82729

% The wheelchart macro
\newcommand{\wheelchart}[3]{
    \def\innerradius{#2}%
    \def\outerradius{#1}%

    % Calculate total
    \pgfmathsetmacro{\totalnum}{0}
    \foreach \value/\colour/\name in {#3} {
        \pgfmathparse{\value+\totalnum}
        \global\let\totalnum=\pgfmathresult
    }

    \begin{tikzpicture}

      % Calculate the thickness and the middle line of the wheel
      \pgfmathsetmacro{\wheelwidth}{\outerradius-\innerradius}
      \pgfmathsetmacro{\midradius}{(\outerradius+\innerradius)/2}

      % Rotate so we start from the top
      \begin{scope}[rotate=90]

      % Loop through each value set. \cumnum keeps track of where we are in the wheel
      \pgfmathsetmacro{\cumnum}{0}
      \foreach \value/\colour/\name in {#3} {
            \pgfmathsetmacro{\newcumnum}{\cumnum + \value/\totalnum*360}

            % Calculate the percent value
            \pgfmathsetmacro{\percentage}{\value/\totalnum*100}
            % Calculate the mid angle of the colour segments to place the labels
            \pgfmathsetmacro{\midangle}{-(\cumnum+\newcumnum)/2}

            % This is necessary for the labels to align nicely
            \pgfmathparse{
               (-\midangle<180?"west":"east")
            } \edef\textanchor{\pgfmathresult}
            \pgfmathsetmacro\labelshiftdir{1-2*(-\midangle>180)}

            % Draw the color segments. Somehow, the \midrow units got lost, so we add 'pt' at the end. Not nice...
            \fill[\colour] (-\cumnum:\outerradius) arc (-\cumnum:-(\newcumnum):\outerradius) --
            (-\newcumnum:\innerradius) arc (-\newcumnum:-(\cumnum):\innerradius) -- cycle;

            % Draw the data labels with percentages
            % \draw  [*-,thin] node [append after command={(\midangle:\midradius pt) -- (\midangle:\outerradius + 1ex) -- (\tikzlastnode)}] at (\midangle:\outerradius + 1ex) [xshift=\labelshiftdir*0.5cm,inner sep=0pt, outer sep=0pt, ,anchor=\textanchor]{\name: \pgfmathprintnumber{\percentage}\%};
            % Draw the data labels without percentages
            \draw  [*-,thin] node [append after command={(\midangle:\midradius pt) -- (\midangle:\outerradius + 1ex) -- (\tikzlastnode)}] at (\midangle:\outerradius + 1ex) [xshift=\labelshiftdir*0.5cm,inner sep=0pt, outer sep=0pt, ,anchor=\textanchor]{\name};

            % Set the old cumulated angle to the new value
            \global\let\cumnum=\newcumnum
        }

      \end{scope}
    \end{tikzpicture}
}

%-------------------------------------------------------------------------------
%   SECTION TITLE
%-------------------------------------------------------------------------------
\cvsection{Work Life Balance}

%-------------------------------------------------------------------------------
%   CONTENT
%-------------------------------------------------------------------------------
\begin{cventries}

% \wheelchart{1.5cm}{0.5cm}{%
%   10/13em/accent!30/Sleeping \& dreaming about work,
%   25/9em/accent!60/Public resolving issues with Yahoo!\ investors,
%   5/12em/accent!10/New York \& San Francisco Ballet Jawbone board member,
%   20/12em/accent!40/Spending time with family,
%   5/8em/accent!20/Business development for Yahoo!\ after the Verizon acquisition,
%   30/9em/accent/Showing Yahoo!\ employees that their work has meaning,
%   5/8em/accent!20/Baking cupcakes
% }

\definecolor{redbird}{HTML}{DA291C}

% Usage: \wheelchart{<value1>/<colour1>/<label1>, ...}
\wheelchart{1.7cm}{1.1cm}{
    15/redbird/\descriptionstyle{Sleeping \& Dreaming about work},
    23/redbird!80/\descriptionstyle{Brainstorming with my team},
    17/redbird!60/\descriptionstyle{Design and Architecture},
    14/redbird!40/\descriptionstyle{Prototyping, development},
    10/redbird!30/\descriptionstyle{News reading, Technology Watch},
    15/redbird!20/\descriptionstyle{Spending time with family, playing Lego with my son},
    5/redbird!10/\descriptionstyle{Cooking}
}

%---------------------------------------------------------
\end{cventries}
